\documentclass[UTF8]{ctexart}
    \title{算法分析与设计基础作业4}
    \author{软件71 骆炳君 2017013573}
    \date{\today}
\usepackage{amsmath}
\usepackage{amssymb}
\usepackage{listings}
\usepackage{xcolor}
\lstset{
    numbers=left, 
    numberstyle= \small, 
    escapeinside=``, % 英文分号中可写入中文
    xleftmargin=2em,xrightmargin=2em, aboveskip=1em,
    framexleftmargin=2em
} 
\setcounter{secnumdepth}{0}
\allowdisplaybreaks[1]
\begin{document}
\maketitle
\pagenumbering{arabic}

\subsection{1.}
记中位数点的位置为$x$,不妨设$\alpha\le\frac{1}{2}\le1-\alpha$,考虑$P_0=P(x\le\alpha\le\frac{1}{2})$.因为仅考虑划分比例,可认为数组A是无限长的,即选出三个元素的过程是相互独立的,同时因为选出的三个元素的位置在所有位置中是等可能的,所以可应用几何概型,将$P_0$分为两种情况:有且仅有2个元素落在前$\alpha$区间、三个元素均落在前$\alpha$区间.

则$$P_0=\alpha^3+\binom{3}{2}\alpha^2(1-\alpha)=3\alpha^2-2\alpha^3$$

由划分的对称性可得,所求概率为$1-2P_0=1-6\alpha^2+4\alpha^3$

\subsection{2.}
\subsubsection{a.}
此时每一次划分都将数组分为$(n-1):1$,即快速排序的最坏情况,$T(n)=\Theta(n^2)$.

\subsubsection{b.}
\begin{lstlisting}
    x=A[r]
    i=p-1
    j=p-1
    for k=p to r
        if A[k]<x
            i=i+1
            j=j+1
            exchange A[k] with A[j]
            exchange A[j] with A[i]
        elif A[k]=x
            j=j+1
            exchange A[k] with A[j]
    return i+1,j
\end{lstlisting}

\subsubsection{c.}
\lstset{language=Python}
\begin{lstlisting}
    import random

    def PARTITION(A, p, r):
        x = A[r]
        i = p-1
        j = p-1
        for k in range(p, r+1):
            if A[k] < x:
                i = i+1
                j = j+1
                A[k], A[j] = A[j], A[k]
                A[j], A[i] = A[i], A[j]
            elif A[k] == x:
                j = j+1
                A[k], A[j] = A[j], A[k]
        return i+1, j


    def RANDOMIZED_PARTITON(A, p, r):
        i = random.randint(p, r)
        A[r], A[i] = A[i], A[r]
        return PARTITION(A, p, r)


    def QUICKSORT(A, p, r):
        if p < r:
            q, t = RANDOMIZED_PARTITON(A, p, r)
            QUICKSORT(A, p, q - 1)
            QUICKSORT(A, t + 1, r)
\end{lstlisting}

\subsubsection{d.}
应该改变(7.2)以后的分析方法,在算法实现过程中可以将与x相等的元素划分到小于x的集合中且不改变时间复杂度,这样即使没有互异的假设,也能保证一旦一个满足$z_i\le x<z_j$的主元被选定后,所有这样的$z_i$、$z_j$都不会相互比较了,因此不会影响其后的分析和结论.
\end{document}