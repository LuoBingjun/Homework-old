\documentclass[UTF8]{ctexart}
    \title{算法分析与设计基础作业1}
    \author{软件71 骆炳君 2017013573}
    \date{\today}
\usepackage{amsmath}
\usepackage{amssymb}
\allowdisplaybreaks[1]
\begin{document}
\maketitle
\pagenumbering{arabic}

\paragraph{a.}
\begin{align*}
    &\forall f(n)=\Theta(n^2)\\
    &\exists c_1,c_2,n_0>0,s.t.\forall n>n_0,0\le c_1n^2\le f(n)\le c_2n^2\\
    &\therefore c_1n^2\le 2n+f(n) \le c_2n^2+2n\\
    &\text{取}c_3>c_2,n_1=max\{n_0,\frac{2}{c_3-c_2}\}\\
    &\because\exists c_1,c_3,n_1>0,s.t.\forall n>n_1,0\le c_1n^2\le 2n+f(n)\le c_2n^2+2n\le c_3n^2\\
    &\therefore 2n+f(n)=\Theta(n^2)\\
    &\forall f(n)=\Theta(n^2)\\
    &\exists c_1,c_2,n_0>0,s.t.\forall n>n_0,0\le c_1n^2\le f(n)\le c_2n^2\\
    &\therefore c_1n^2-2n\le f(n)-2n \le c_2n^2\\
    &\text{取}c_3<c_1,n_1=max\{n_0,\frac{2}{c_1-c_3}\}\\
    &\because\exists c_2,c_3,n_1>0,s.t.\forall n>n_1,0\le c_3n^2\le c_1n^2-2n\le f(n)-2n\le c_2n^2\\
    &\therefore f(n)=\Theta(n^2)+2n\\
    &\text{综上可证:}\Theta(n^2)+2n=\Theta(n^2)
\end{align*}

\paragraph{b.}
\begin{align*}
    &\forall f(n)\in \Theta(g(n)),\\
    &\because\exists c_1,c_2,n_0>0,s.t.\forall n>n_0,0\le c_1g(n)\le f(n)\le c_2g(n)\\
    &\therefore \lim_{n\rightarrow\infty}\frac{f(n)}{g(n)}\ne0\\
    &\therefore f(n)\notin o(g(n))\\
    &\text{即}\Theta(g(n))\cap o(g(n))=\emptyset
\end{align*}

\paragraph{c.}
\begin{align*}
    &\text{欲证明原命题,只需证明}\\
    &\exists f(n),g(n),\text{满足}f(n)\in O(g(n)),f(n)\notin\Theta(g(n))\cup o(g(n))\\
    &\text{取}g(n)=n^3,f(n)=\begin{cases}
        n^3,&\text{n为奇数}\\
        n^2,&\text{n为偶数}
    \end{cases}\\
    &\text{满足这个条件,所以原命题成立}
\end{align*}

\paragraph{d.}
\begin{align*}
    &\text{若}f(n)\ge g(n)\\
    &\text{则}f(n)=max\{f(n),g(n)\}\\
    &\therefore \forall n>0, \frac{1}{2}(f(n)+g(n))\le f(n)\le f(n)+g(n)\\
    &\therefore f(n)=\Theta(f(n)+g(n))\\
    &\text{若}f(n)<g(n)\\
    &\text{则}g(n)=max\{f(n),g(n)\}\\
    &\therefore \forall n>0, \frac{1}{2}(f(n)+g(n))\le g(n)\le f(n)+g(n)\\
    &\therefore g(n)=\Theta(f(n)+g(n))\\
    &\text{综上可证:}max\{f(n),g(n)\}=\Theta(f(n)+g(n))
\end{align*}

\paragraph{3-3}
\subparagraph{(a)}
排列如下:

$2^{2^{n+1}}$

$2^{2^n}$

$(n+1)!$

$n!$

$e^n$

$n\cdot2^n$

$2^n$

$(\frac{3}{2})^n$

$(lgn)^{lgn}$

$n^{lglgn}$

$(lgn)!$

$n^3$

$4^{lgn}$

$n^2$

$nlgn$

$lg(n!)$

$n$

$2^{lgn}$

$(\sqrt{2})^{lgn}$

$2^{\sqrt{2lgn}}$

$lg^2n$

$lnn$

$\sqrt{lgn}$

$lnlnn$

$2^{lg^*n}$

$lg^*n$

$lg^*(lgn)$

$lg(lg^*n)$

$n^{1/lgn}$

$1$

划分的等价类为:

$\{2^{2^{n+1}}\}$

$\{2^{2^n}\}$

$\{(n+1)!\}$

$\{n!\}$

$\{e^n\}$

$\{n\cdot2^n\}$

$\{2^n\}$

$\{(\frac{3}{2})^n\}$

$\{(lgn)^{lgn},n^{lglgn}\}$

$\{(lgn)!\}$

$\{n^3\}$

$\{4^{lgn},n^2\}$

$\{nlgn,lg(n!)\}$

$\{n,2^{lgn}\}$

$\{(\sqrt{2})^{lgn}\}$

$\{2^{\sqrt{2lgn}}\}$

$\{lg^2n\}$

$\{lnn\}$

$\{\sqrt{lgn}\}$

$\{lnlnn\}$

$2^{lg^*n}$

$\{lg^*n\}$

$\{lg^*(lgn)\}$

$\{lg(lg^*n)\}$

$\{n^{1/lgn},1\}$

\subparagraph{(b)}
$$
f(n)=
\begin{cases}
    0,&n\text{为奇数}\\
    2^{2^{n+2}},&n\text{为偶数}
\end{cases}
$$
\end{document}