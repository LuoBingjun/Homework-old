\documentclass[]{article}
\usepackage{lmodern}
\usepackage{amssymb,amsmath}
\usepackage{ifxetex,ifluatex}
\usepackage{fixltx2e} % provides \textsubscript
\ifnum 0\ifxetex 1\fi\ifluatex 1\fi=0 % if pdftex
  \usepackage[T1]{fontenc}
  \usepackage[utf8]{inputenc}
\else % if luatex or xelatex
  \ifxetex
    \usepackage{mathspec}
  \else
    \usepackage{fontspec}
  \fi
  \defaultfontfeatures{Ligatures=TeX,Scale=MatchLowercase}
\fi
% use upquote if available, for straight quotes in verbatim environments
\IfFileExists{upquote.sty}{\usepackage{upquote}}{}
% use microtype if available
\IfFileExists{microtype.sty}{%
\usepackage{microtype}
\UseMicrotypeSet[protrusion]{basicmath} % disable protrusion for tt fonts
}{}
\usepackage[margin=1in]{geometry}
\usepackage{hyperref}
\hypersetup{unicode=true,
            pdftitle={大作业报告},
            pdfauthor={骆炳君 2017013573 软件71},
            pdfborder={0 0 0},
            breaklinks=true}
\urlstyle{same}  % don't use monospace font for urls
\usepackage{color}
\usepackage{fancyvrb}
\newcommand{\VerbBar}{|}
\newcommand{\VERB}{\Verb[commandchars=\\\{\}]}
\DefineVerbatimEnvironment{Highlighting}{Verbatim}{commandchars=\\\{\}}
% Add ',fontsize=\small' for more characters per line
\usepackage{framed}
\definecolor{shadecolor}{RGB}{248,248,248}
\newenvironment{Shaded}{\begin{snugshade}}{\end{snugshade}}
\newcommand{\AlertTok}[1]{\textcolor[rgb]{0.94,0.16,0.16}{#1}}
\newcommand{\AnnotationTok}[1]{\textcolor[rgb]{0.56,0.35,0.01}{\textbf{\textit{#1}}}}
\newcommand{\AttributeTok}[1]{\textcolor[rgb]{0.77,0.63,0.00}{#1}}
\newcommand{\BaseNTok}[1]{\textcolor[rgb]{0.00,0.00,0.81}{#1}}
\newcommand{\BuiltInTok}[1]{#1}
\newcommand{\CharTok}[1]{\textcolor[rgb]{0.31,0.60,0.02}{#1}}
\newcommand{\CommentTok}[1]{\textcolor[rgb]{0.56,0.35,0.01}{\textit{#1}}}
\newcommand{\CommentVarTok}[1]{\textcolor[rgb]{0.56,0.35,0.01}{\textbf{\textit{#1}}}}
\newcommand{\ConstantTok}[1]{\textcolor[rgb]{0.00,0.00,0.00}{#1}}
\newcommand{\ControlFlowTok}[1]{\textcolor[rgb]{0.13,0.29,0.53}{\textbf{#1}}}
\newcommand{\DataTypeTok}[1]{\textcolor[rgb]{0.13,0.29,0.53}{#1}}
\newcommand{\DecValTok}[1]{\textcolor[rgb]{0.00,0.00,0.81}{#1}}
\newcommand{\DocumentationTok}[1]{\textcolor[rgb]{0.56,0.35,0.01}{\textbf{\textit{#1}}}}
\newcommand{\ErrorTok}[1]{\textcolor[rgb]{0.64,0.00,0.00}{\textbf{#1}}}
\newcommand{\ExtensionTok}[1]{#1}
\newcommand{\FloatTok}[1]{\textcolor[rgb]{0.00,0.00,0.81}{#1}}
\newcommand{\FunctionTok}[1]{\textcolor[rgb]{0.00,0.00,0.00}{#1}}
\newcommand{\ImportTok}[1]{#1}
\newcommand{\InformationTok}[1]{\textcolor[rgb]{0.56,0.35,0.01}{\textbf{\textit{#1}}}}
\newcommand{\KeywordTok}[1]{\textcolor[rgb]{0.13,0.29,0.53}{\textbf{#1}}}
\newcommand{\NormalTok}[1]{#1}
\newcommand{\OperatorTok}[1]{\textcolor[rgb]{0.81,0.36,0.00}{\textbf{#1}}}
\newcommand{\OtherTok}[1]{\textcolor[rgb]{0.56,0.35,0.01}{#1}}
\newcommand{\PreprocessorTok}[1]{\textcolor[rgb]{0.56,0.35,0.01}{\textit{#1}}}
\newcommand{\RegionMarkerTok}[1]{#1}
\newcommand{\SpecialCharTok}[1]{\textcolor[rgb]{0.00,0.00,0.00}{#1}}
\newcommand{\SpecialStringTok}[1]{\textcolor[rgb]{0.31,0.60,0.02}{#1}}
\newcommand{\StringTok}[1]{\textcolor[rgb]{0.31,0.60,0.02}{#1}}
\newcommand{\VariableTok}[1]{\textcolor[rgb]{0.00,0.00,0.00}{#1}}
\newcommand{\VerbatimStringTok}[1]{\textcolor[rgb]{0.31,0.60,0.02}{#1}}
\newcommand{\WarningTok}[1]{\textcolor[rgb]{0.56,0.35,0.01}{\textbf{\textit{#1}}}}
\usepackage{longtable,booktabs}
\usepackage{graphicx,grffile}
\makeatletter
\def\maxwidth{\ifdim\Gin@nat@width>\linewidth\linewidth\else\Gin@nat@width\fi}
\def\maxheight{\ifdim\Gin@nat@height>\textheight\textheight\else\Gin@nat@height\fi}
\makeatother
% Scale images if necessary, so that they will not overflow the page
% margins by default, and it is still possible to overwrite the defaults
% using explicit options in \includegraphics[width, height, ...]{}
\setkeys{Gin}{width=\maxwidth,height=\maxheight,keepaspectratio}
\IfFileExists{parskip.sty}{%
\usepackage{parskip}
}{% else
\setlength{\parindent}{0pt}
\setlength{\parskip}{6pt plus 2pt minus 1pt}
}
\setlength{\emergencystretch}{3em}  % prevent overfull lines
\providecommand{\tightlist}{%
  \setlength{\itemsep}{0pt}\setlength{\parskip}{0pt}}
\setcounter{secnumdepth}{0}
% Redefines (sub)paragraphs to behave more like sections
\ifx\paragraph\undefined\else
\let\oldparagraph\paragraph
\renewcommand{\paragraph}[1]{\oldparagraph{#1}\mbox{}}
\fi
\ifx\subparagraph\undefined\else
\let\oldsubparagraph\subparagraph
\renewcommand{\subparagraph}[1]{\oldsubparagraph{#1}\mbox{}}
\fi

%%% Use protect on footnotes to avoid problems with footnotes in titles
\let\rmarkdownfootnote\footnote%
\def\footnote{\protect\rmarkdownfootnote}

%%% Change title format to be more compact
\usepackage{titling}

% Create subtitle command for use in maketitle
\providecommand{\subtitle}[1]{
  \posttitle{
    \begin{center}\large#1\end{center}
    }
}

\setlength{\droptitle}{-2em}

  \title{大作业报告}
    \pretitle{\vspace{\droptitle}\centering\huge}
  \posttitle{\par}
    \author{骆炳君 2017013573 软件71}
    \preauthor{\centering\large\emph}
  \postauthor{\par}
      \predate{\centering\large\emph}
  \postdate{\par}
    \date{2019/6/28}


\begin{document}
\maketitle

\subsection{背景描述}

EEC(欧洲经济共同体)和COMECON(经济互助委员会)是冷战时期欧洲最主要的两大经济合作组织,涵盖了苏联、英国、法国、东德、西德等大多数欧洲经济体。

本研究将利用EEC和COMECON国家的蛋白质消费量数据,分析不同种类的蛋白质消费量间的关系,并找出两大经济合作组织在蛋白质消费上的差异。

\subsection{数据}

\subsubsection{介绍}

本研究使用的Europe Protein
consumption数据集,记录了25个主要欧洲国家对肉、蛋、奶、鱼等9种类型的蛋白质消费量情况,其中包括16个EEC国家和9个COMECON国家。由于Country项在本研究中没有意义,因此不考虑本列数据。

\begin{longtable}[]{@{}llrrrrrrrrr@{}}
\caption{数据集概览}\tabularnewline
\toprule
Economy & Country & RedMeat & WhiteMeat & Eggs & Milk & Fish & Cereals &
Starch & Nuts & Fr.Veg\tabularnewline
\midrule
\endfirsthead
\toprule
Economy & Country & RedMeat & WhiteMeat & Eggs & Milk & Fish & Cereals &
Starch & Nuts & Fr.Veg\tabularnewline
\midrule
\endhead
C & Albania & 10.1 & 1.4 & 0.5 & 8.9 & 0.2 & 42.3 & 0.6 & 5.5 &
1.7\tabularnewline
E & Austria & 8.9 & 14.0 & 4.3 & 19.9 & 2.1 & 28.0 & 3.6 & 1.3 &
4.3\tabularnewline
\bottomrule
\end{longtable}

\hypertarget{eda}{%
\subsubsection{EDA}\label{eda}}

进行探索性数据分析,首先得到下面的分析图:

观察上图,可以初步得出以下结论:

\begin{itemize}
\tightlist
\item
  红肉与奶,蛋与红肉、白肉、奶,坚果与谷物的消费量存在着较强的正相关关系,谷物与蛋、奶、鱼、淀粉,坚果与白肉、蛋、奶的消费量间存在着较强的负相关关系,蔬果与其他种类是食品的消费量都没有明显的相关关系。
\item
  EEC国家的红肉、蛋、奶、鱼的消费量显著高于COMECON国家,COMECON国家的谷物、坚果的消费量显著高于EEC国家,EEC国家的蔬果消费量的方差显著大于COMECON国家的蔬果消费量。
\end{itemize}

\subsubsection{预处理}

对数据进行初步处理,对Economy项进行数值化(C对应1,E对应2),以方便后续的研究。

\subsection{建模}

\subsubsection{因子分析}

\begin{Shaded}
\begin{Highlighting}[]
\KeywordTok{library}\NormalTok{(}\StringTok{"stats"}\NormalTok{)}
\NormalTok{respc <-}\StringTok{ }\KeywordTok{princomp}\NormalTok{(data)}
\NormalTok{respc}
\end{Highlighting}
\end{Shaded}

\begin{verbatim}
## Call:
## princomp(x = data)
## 
## Standard deviations:
##     Comp.1     Comp.2     Comp.3     Comp.4     Comp.5     Comp.6 
## 12.2126359  5.4287362  3.8785887  2.8265815  1.8677511  1.5367332 
##     Comp.7     Comp.8     Comp.9    Comp.10 
##  1.2245192  0.8290467  0.4979834  0.1717014 
## 
##  10  variables and  25 observations.
\end{verbatim}

\hypertarget{section}{%
\subsubsection{}\label{section}}

\subsection{改进与不足}

\subsection{附录}

\begin{verbatim}
data <- read.table('Europrotein.dat',header = TRUE,sep = ':')

# EDA
library("ggplot2")
library("GGally")
ggpairs(subset(data, select = -Country ))
\end{verbatim}


\end{document}
